\documentclass[12pt,a4paper]{article}

% XeLaTeX 中文設置
\usepackage{fontspec}
\usepackage{xeCJK}
\setmainfont{Times New Roman}
\setCJKmainfont{Noto Serif CJK TC}

% 頁面與格式設定(近似 NEJM)
\usepackage[top=2.5cm, bottom=2.5cm, left=3cm, right=3cm]{geometry}
\usepackage{setspace}
\usepackage{titlesec}
\usepackage{fancyhdr}
\usepackage{graphicx}
\usepackage{caption}

% 行距
\setstretch{1.5}

% 標題設定
\titleformat*{\section}{\large\bfseries}
\titleformat*{\subsection}{\normalsize\bfseries}
\titleformat*{\subsubsection}{\normalsize}

% 頁首頁尾
\pagestyle{fancy}
\fancyhf{}
\fancyhead[L]{NEJM-style Academic Paper}
\fancyhead[R]{\thepage}

% 參考文獻
\usepackage[backend=biber, style=numeric, sorting=none]{biblatex}
\addbibresource{refs.bib}

% 摘要環境
\newenvironment{abstract}{
  \begin{center}
    \bfseries Abstract
  \end{center}
  \begin{quote}
}{\end{quote}}

\begin{document}

\begin{center}
  {\LARGE \textbf{論文標題 Title in English}}\\[1.5em]
  {\large 作者1\textsuperscript{1},作者2\textsuperscript{2}}\\[1em]
  {\small
    \textsuperscript{1}單位1 \\
    \textsuperscript{2}單位2 \\
    \texttt{email@example.com}
  }
\end{center}

\begin{abstract}
這裡撰寫中文摘要。\\
\textbf{Background:} ...\\
\textbf{Methods:} ...\\
\textbf{Results:} ...\\
\textbf{Conclusions:} ...
\end{abstract}

\section*{Introduction}
% Introduction section
本節為簡要介紹研究背景與目的。


% 根據論文型態 include(請依需求選擇)
% % Original Article Content
\section*{Methods}
請於此處撰寫研究方法。

\section*{Results}
請於此處撰寫研究結果。

% % Case Report Content
\section*{Case Presentation}
請於此處撰寫病例報告內容。

\section*{Discussion}
請於此處撰寫討論。

% % Medical Image Content
\section*{Image Description}
請於此處撰寫醫學影像案例描述。


\section*{Discussion}
% Discussion section
本節可補充討論與結論。


\section*{References}
\printbibliography

\end{document}
